\documentclass[a4paper]{report}

%% Language and font encodings
\usepackage[english]{babel}
\usepackage[utf8x]{inputenc}
\usepackage[T1]{fontenc}

%% Sets page size and margins
\usepackage[a4paper,top=3cm,bottom=2cm,left=3cm,right=3cm,marginparwidth=1.75cm]{geometry}

%% Useful packages
\usepackage{amsmath}
\usepackage{graphicx}
\usepackage[colorinlistoftodos]{todonotes}
\usepackage[colorlinks=true, allcolors=blue]{hyperref}

\title{Change Detection for Digital Art}
\author{Joseph Allen}

\begin{document}
\maketitle

\begin{abstract}
In Computer Science we are used to collaboration, it is a fundamental aspect of programming careers to apply technical skills with a real world problem to create a useful application or system. Art and Computer Science are almost as far apart as possible, where Art strives to create arguably meaningless products with no use or application other than to convey or elicit emotion. 

The outcome of this project will be the creation of a product which will allow artists to trivially perform basic change detection with the sole purpose of artistic play. This project will help introduce non-technical artists to programming, inspire a sense of curiosity to delve deeper into the code, and hopefully transition them into mastery of a language.

My goal here is to create a project the encourages Artists to be more like Computer Scientist, and vice-versa.
\end{abstract}

\renewcommand{\abstractname}{Acknowledgements}
\begin{abstract}
 Thanks Mum!
\end{abstract}

\tableofcontents

\section{Introduction}
\subsection{The Current Problem}
\subsection{Proposed Solution}
\subsection{What is Processing?}
\subsection{How do we currently do things?}
\subsection{Aims}
\subsection{Methodology}

\section{Background RENAME}
\subsection{Conception}
\subsubsection{Sarah (Client)}
\subsubsection{Code as Art}
\subsubsection{Processing as inspiration}
\subsection{SAKBOT}

\section{Change Detection Basics}
\subsection{Thresholding}
\subsection{Euclidean Distance between two points}
\subsection{Color Normalization}
\subsection{HSB Color Space}

\section{Change Detection Advanced}
\subsection{OpenCV}
\subsubsection{Built in diff}
\subsubsection{Built in background subtraction}
\subsubsection{Erosion}
\subsubsection{Dilation}
\subsubsection{Optical Flow}
\subsubsection{Blob Detection}

\section{Building an Artistic Product}
\subsection{Requirements Elicitation}
\subsubsection{Types of Interview}
\subsubsection{decision and justification}
\subsection{Codification}
\subsection{Requirements}

\section{Functionality}
\subsection{Camera}
\subsection{Color}
\subsection{Image}
\subsection{Video}
\subsection{Code}

\section{Usability}
\subsection{Read Me}
\subsection{Tutorials}

\section{Evaluation}
\subsection{Expectations}
\subsection{Evolution of project}
\subsection{What I have learned}


\bibliographystyle{alpha}
\bibliography{sample}

\end{document}